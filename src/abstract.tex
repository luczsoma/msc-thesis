\begin{otherlanguage}{magyar}

\paragraph*{Kivonat}
\phantomsection
\addcontentsline{toc}{chapter}{Kivonat}
\thispagestyle{plain}

Napjaink globalizált világának működésében kulcsfontosságú szerepet tölt be a diplomácia. Diplomatává válni hosszú folyamat, mely korai elhivatottságot kíván – gyakran középiskolás vagy egyetemista korban, világszervezetek munkájának tanulmányi célú szimulációjában való részvétellel kezdődik egy karrier. \emph{Egy} leendő diplomata karrierjét támogatva nemcsak betekintést nyújthatunk az általa is formált közös jövőnkbe, de hosszú távon annak alakításában is részt vehetünk. \emph{Az összes} leendő diplomata karrierjét tekintve a lehetőségek tárháza határtalan, az ezzel járó felelősség pedig hatalmas.

A \emph{Model United Nations (MUN)} keretrendszerben világszerte évente többszázas nagyságrendben megrendezett konferenciákon résztvevő középiskolás és egyetemista diákok az Egyesült Nemzetek Szervezete (ENSZ) mindennapi munkájának formális szimulációján keresztül tanulhatnak diplomáciáról, nemzetközi kapcsolatokról, világpolitikáról – kockázatmentes, tényekre és információkra alapozott vitakultúrát kultiváló környezetben, gyakran tapasztalt karrierdiplomaták támogatásával.

A világ MUN-közösségének összefogására több szoftveres kísérlet is született. Ezek többnyire egy-egy problémára igyekeznek elszigetelt megoldást adni, így kapcsolatépítésre, konferenciák szakmai szervezésére, illetve rendezvények általános adminisztrációjára eltérő – gyakran házon belüli – szoftverek használatosak. Ezen alkalmazások nem kötik össze a közösség egészét, és nem adnak teljes megoldást az adminisztratív problémákra sem.

Dolgozatomban kifejtem, ahogyan megtervezem, lefejlesztem, és webalkalmazásként publikusan elérhetővé teszem a \emph{Diplomatiq} nevű, MUN-konferenciák szervezésére alkalmas közösségi hálózatot. A \emph{Diplomatiq} hosszú távú célkitűzése az, hogy a diplomaták elsődleges közösségi platformjaként nyújtson integrált megoldást az MUN-világban felmerülő adminisztratív problémákra.

A tervezés és fejlesztés teljes folyamata alatt fókuszban tartottam két alapvető szempontot. Az első szempont, hogy a rendszer „használatra kész” minőségben készüljön el, és később igény szerint bővíthető legyen további közösségi, adminisztratív, illetve valós idejű adatelemzési funkcionalitással. Ennek célja, hogy a szoftver a jövőben az MUN-szcénán kívül valódi diplomáciai alkalmazásokban is helyt tudjon állni. A második szempont – a tárolt személyes adatok, illetve a szoftver leendő alkalmazási lehetőségeinek figyelembe vételével – az, hogy a rendszer már a kezdetektől modern, réteges, kriptográfiai biztosítékokat nyújtó biztonsági architektúrára alapozva készüljön el.

A rendszer tervezése és fejlesztése során a mérnöki szempontokon felül arra is figyelmet fordítottam, hogy a \emph{Diplomatiq}, mint majdani vállalkozás az elvégzett munkámra egyszerűen ráépíthető legyen. A szoftver lefejlesztéséhez és publikációjához szükséges előfizetéseket, szolgáltatásokat és rendszereket mind olyan megfontoltsággal választottam ki és integráltam, mintha egy vállalkozást indítanék el. Dolgozatomban az ezzel kapcsolatban felmerülő adminisztratív és pénzügyi teendők mellett a rendszer egy kezdetleges üzleti modelljéről is beszámolok – kisebb terjedelemben, mérnöki diplomatervről lévén szó.

\end{otherlanguage}

\clearpage

\paragraph*{Abstract}
\phantomsection
\addcontentsline{toc}{chapter}{Abstract}
\thispagestyle{plain}

Diplomacy plays a key role in the operation of today's globalized world. Turning into a diplomat is a long process and involves early dedication — careers often start in high schools or universities, by students taking part in academic simulations of various intergovernmental organizations' work. Supporting \emph{a} prospective diplomat's career not only enables us to peek into the future through them, but in the long run, we can also take part in jointly shaping tomorrow's world. Considering \emph{all} prospective diplomats' careers, the possibilities are endless, and the associated responsibility is immense.

The world of junior diplomats mostly consists of conferences — annually hundreds of them, worldwide — held within the framework of the \emph{Model United Nations (MUN)}. During these events, high school and university students formally simulate the everyday work of the United Nations (UN), which enables them to learn about diplomacy, international relations and world politics — in a risk-free environment, cultivating debates based solely on facts and information. These conferences are often attended by experienced senior diplomats as well, with the goal of supporting and educating the future generation.

There are several software-involved attempts for bringing together the MUN community. Most of these attempts solve one isolated problem of the collective at a time: social networking, organizing the professional part of conferences, and administering the actual events usually involves several different — mostly in-house — software. These applications neither link the community together, nor do they offer a complete solution to administrative problems.

In this thesis I design, implement and publish \emph{Diplomatiq}, a social network software system for diplomats, suitable for organizing MUN conferences. The long-term goal of \emph{Diplomatiq} is to provide an integrated solution for administrative problems in the MUN world, while being the sole professional networking platform for its diplomat users.

During the whole process of the design and implementation, I focused on two key points. The first point is that the system should be implemented in production-grade quality, and it should be easily extendable with further social, administrative, and real-time data analytics features as needed. The goal of this is to enable the system to cover the needs of real-world diplomatic applications as well, outside the MUN scene. The second point — considering the stored personal information, and the future applications — is that the system should be implemented upon a modern, layered security architecture, which provides cryptographic assurances in terms of application and data security.

Besides engineering aspects, I also paid attention to being able to build \emph{Diplomatiq} as a prospective company upon my work. Subscriptions, services and systems needed for the implementation and publication were chosen and integrated with the same amount of consideration as I was starting company. In this thesis I present the related administrative and financial aspects of this too, as well as a primitive business model — briefly only, this being an engineering thesis.

%%% Local Variables:
%%% mode: latex
%%% TeX-master: "main"
%%% End:
