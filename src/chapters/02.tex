\chapter{Preliminaries and related work}
\label{chapter:preliminaries}

This chapter summarizes the preliminary knowledge needed for a high-level understanding of this thesis. It details the concept of Model United Nations and my personal experience with MUN. It also defines the idea of a social network. Then it gives examples for domain-specific social networks, and presents existing software solutions for the MUN community.

\section{The Model United Nations framework}
\label{section:munframework}

\subsection{Introduction}

The Model United Nations (MUN) framework is an academic simulation of the everyday operation of the United Nations (UN). It is typically an extra-curricular activity materializing as annual, few-day-long conferences organized by students of high schools or universities. Participants welcomed from all over the world take on the roles of assigned nations' UN delegates, forming diplomatic delegations with their peers. There are hundreds of such conferences taking place every year~\cite{mymunconferencelist}. While a medium-size conference has a few hundred participants and another few hundreds of organizers, the current largest Model UN conference — The Hague International Model United Nations (THIMUN) — attracts over 3,200 students from around 200 schools, from more than 100 different countries~\cite{thimunabout}.

Delegates are placed in UN-like committees, where they discuss topics related to international issues and conflicts by the methods of moderated formal debate. The conduct of the debates and the conference is specified in the \emph{Rules of Procedure}, a conference-specific, formal regulation derived from a similar document of the United Nations~\cite{unmunrop}. The result of the debate is a UN-like \emph{resolution}: a formal document expressing the opinion or will of a committee. Resolutions are generally recommendations, but in some cases — like in case of a resolution adopted by the Security Council: the UN body with \textquote{primary responsibility for the maintenance of international peace and security}~\cite{uncharter} — the adopted resolution is legally binding for all member states. Although MUN resolutions of course are never legally binding, larger MUN conferences like THIMUN forward their adopted resolutions to the UN. These forwarded MUN resolutions are occasionally formulated into real-world UN resolutions after further debate and amendments.

The country and committee assignments are known in advance, which enables delegates to perform research and develop their positions before the conference. Students usually build their stances upon the actual standpoints of the countries they represent, but this is not a requirement. Since usually all positions of a given country's diplomatic delegation is assigned to students arriving from the same school, delegates representing the same country can construct complex nationwide strategies across different committees by cooperating with each other in advance.

The larger a conference is, the more possibilities it has regarding the simulation of the actual workings of the UN, or — as the community reinvents itself — even other intergovernmental bodies, like subsidiaries of the European Union. Even though the UN has only 193 member states~\cite{unmembers}, simultaneously simulating the whole operation of the six main organs\footnote{The six main organs of the UN are: the General Assembly with several subsidiary boards, commissions, committees, councils, panels, working groups and others~\cite{gasubsidiaries}; the Security Council; the Economic and Social Council; the Trusteeship Council; the International Court of Justice; and the Secretariat~\cite{unmainorgans}.}, and the Secretariat of the UN requires much more participants.

\subsection{History}

The history of Model United Nations dates back to the early 20\textsuperscript{th} century. The first similar event is believed to have been held in November 1921 by the Oxford International Assembly~\cite{historyofthefirstmun}. It was based on the operation of the League of Nations, the first worldwide intergovernmental organization founded by the Allied powers after the First World War to maintain world peace. Although the League of Nations was formally disbanded in 1946 and its powers were transferred to the United Nations established in 1945, the organization marks an important milestone of intergovernmental cooperation~\cite{leagueofnationsbritannica}.

The first well-documented Model League of Nations conference was organized by the Harvard International Assembly in 1923. It featured the same basic characteristics that modern MUN conferences have: it was organized by an academic institution, had moderated formal debate about international conflicts in committees, and had resolutions adopted as the result of the work conducted during the conference~\cite{historyofthefirstmun}.

The era of Model United Nations started in the 1950s with the establishment of the first high school MUN, Berkeley Model United Nations in 1952, and two other MUNs founded by Harvard University: Harvard Model United Nations in 1953 and Harvard National Model United Nations in 1954. The founding of The Hague International Model United Nations in 1968 led to the global expansion of high school MUN conferences~\cite{top10eventsofmun}. THIMUN was the first MUN in Europe, and is today's largest MUN conference~\cite{thimunabout}. In 1991, the Harvard WorldMUN, a university level MUN rapidly accelerated the spread of university-level MUN. In 2007, the actuation of the BestDelegate.com portal significantly increased the online existence of MUN, providing research and preparatory resources for delegates attending MUN conferences. Founding of MyMUN, an MUN-specific registration and administration system, and MUNPlanet\footnote{At the time of writing this thesis, the website of MUNPlanet~\cite{munplanetwebsite} is not reachable, and its Facebook page~\cite{munplanetfacebook} having more than 150,000 followers received its last update in June 2019.}, an MUN-specific social and knowledge-sharing network furthered the presence of Model United Nations on the Internet~\cite{top10eventsofmun}.

\subsection{MUN in numbers}

I have not found any databases, publications, or studies, which would yield satisfactory statistics about Model United Nations. However, according to several portals, websites and Facebook-pages, we can make assumptions about the worldwide spread of MUN.

\subsubsection{Conferences}

At the time of writing this thesis, MyMUN lists 2,457 conferences from December 2012 until today~\cite{mymunconferencelist}. Calculating with roughly 8 years, and with the broad simplification that there were an equal number of conferences organized every year, there are annually more than 300 conferences worldwide, at least based on solely the data of MyMUN. Considering that MyMUN covers only a small fraction of all MUNs in the world, the number of annual MUN conferences likely goes well into the order of thousands.

\subsubsection{Participants}

MyMUN claims having over 100,000 registered members and over 900,000 yearly visitors~\cite{mymunwebsite}. The Facebook page of MUNPlanet~\cite{munplanetfacebook} has over 150,000 followers. According to a 2007 calculation~\cite{howbigismun}, there are 180,000 MUN participants in the United States only. The BestDelegate.com portal is used by over 750,000 people worldwide~\cite{bestdelegate-about}. Considering the numbers above and MUN's increasing popularity, I would assume that millions of unique students attend Model United Nations conferences every year.

\subsection{Networking within Model United Nations}

MUN conferences offer a number of networking opportunities. First of all, delegates attend committee sessions, where they debate international issues, cooperate in producing resolutions, and leverage simulated international relations along their represented countries' best interests. Therefore the main contact point among them is work: they get to know fellow delegates by observing their leadership, public speaking, and negotiation skills in a competitive field. Committee and lunch breaks enable them to further their acquaintances during the day either professionally or personally.

Professional diplomats attending conferences as guests can also take part in committee sessions as observers, or as actual delegates or chairpersons, but they are more likely to attend the official ceremonies or soirées\footnote{elegant evening party, usually with snacks and drinks}. This way, delegates can interact with professional diplomats without unnecessary formalities of real-world diplomacy. Career diplomats are renowned to have appreciable social skills~\cite{fsicapabilities}, which in this setting helps further mellow the atmosphere, and leads to fruitful conversations between generations.

Besides professional programs like debate sessions and official ceremonies, conferences provide a number of other opportunities for delegates to get acquainted with each other. Events like organized sightseeings and afterwork parties allows building informal bonds alongside professional ones.

Aside from maintaining virtual friendships, members of the MUN community often harmonize their conference participations to meet with their foreign acquaintances. Since there is no suitable networking platform for this scenario, participants from different countries usually keep connected and communicate via general social networks, like Facebook. Due to the lack of Facebook's MUN-specific capabilities, a large part of the networking effect is lost, as delegates are not given automated suggestions on which conferences to attend based either on their circle of friends, or on their previous Model UN experience.

\subsection{Administration of a Model United Nations conference}

The general administration of a conference can be divided to two distinct parts. The \emph{professional division} involves administrative tasks related to the Model United Nations framework itself: composing the discussed topics, assigning countries and committees to members of delegations, conducting the actual debates and formal ceremonies, and other features associated with diplomacy. The professional division encompasses everything inside the simulation, where the participants are in their diplomatic roles. The \emph{organizational division} covers the real-world event outside the simulation: travel and hotel arrangements, meals, conference accessories, merchandise, entertainment, among others. According to my personal organizational experience to be detailed in \Cref{section:personalexperience}, the two divisions are separate responsibilities requiring completely different experience, and thus are best to be kept as isolated as possible.

Similarly to the United Nations, the chief administrative officer of an MUN conference is usually the \emph{Secretary-General}, responsible for organizing, administering and conducting the conference. In case of the aforementioned two-division administrative approach, the Secretary-General is responsible only for the professional division of the conference, and reports to the \emph{Conference Manager} or \emph{Project Manager}, who leads the organizational division, and is responsible for the entire conference. Henceforth I refer to the organizational and professional divisions together as management.

In the following sections I detail the procedure of organizing a medium-size, high school-level MUN conference with the two-division administrative approach, broken down to distinct, preemptive phases. Most of the following is based on my personal experience, but it also contains parts I learned from event planners or other Model United Nations conference organizers.

\subsubsection{Preliminary arrangements}

Once the management of the previous year's session agreed upon their successors, the new management starts negotiations with the headmaster of the organizing school. They settle the dates for the conference, discuss necessary resources the school can provide, and start or continue securing external locations\footnote{External locations are often needed to be secured years in advance.} for larger conference ceremonies involving all participants, which the school usually cannot host.

After the initial negotiations, the management announces the conference's subsequent session in the host school with its date and vacancies in the organizational structure, and starts coordinating interviews with eventual applicants wishing to take part in organizing the conference.

\subsubsection{Participants' application to the conference}

After the professional division agreed on the UN bodies to be simulated, they publish preliminary committee assignments with high-level topics on the conference's website or Facebook page. As the management opens the application for participants, delegations and individual delegates apply to the conference specifying their preferences of country and committee assignments. Since high school students rarely travel abroad alone, a delegation is usually accompanied by a couple of teachers of the applicants' school. The accompanying teachers are recorded into the registration system as well as the students.

The application procedure comes with significant paperwork. The management consisting of junior members usually lack the experience and resources for efficiently and securely processing sensitive personal information\footnote{Registration data usually includes the applicant's name, birthdate, email address, phone number, and home address. It is also common to require the applicant's social security number or even the serial number of their passport for speeding up any immigration-related checks and administration.} of hundreds or thousands of applicants, and a school's IT division cannot be expected to be prepared for developing a suitable system either. Thus it is common that the registration is implemented with a simple online form populating insecure spreadsheets, or a primitive in-house application often making data management only more complex. Ready-to-use software like MyMUN are often not customizable enough to be useful for conferences generally requiring somewhat tailored solutions, which leads to the usage of multiple different applications. This can cause more pain than gain by requiring manual data maintenance in unintegrated systems.

\subsubsection{Accepting or rejecting applications \& finalizing country and committee assignments}

Following the closure of the application procedure, the management decides about the acceptance or rejection of prospective participants. As applicants usually register as part of their school's delegation consisting of multiple delegates, it is common that the application of an entire delegation gets accepted or rejected together.

Whether to accept a delegation's application to the conference is decided by considering various factors: these can include the delegation's level of expertise — often with regards to the reputation of previously attended conferences —, the ability of fulfilling their preferences of represented countries provided at registration, and aspects of diversity, among others. In case of more mature conferences, committees are often classified by their members' expected level of MUN experience. A beginner delegate taking a seat in an expert-level committee is inexpedient, as it sets back the efficiency of the debate, or vice versa, it prevents the delegate from staying on their learning curve.

Accepting delegations happens in the interest of filling committee seats — predefined seats of countries represented in committees — in the best possible combination with regards to expertise, personal preference, diversity, and the prospective efficiency of the committee's work. Creating the final assignments of delegations and country-committee pairs can be challenging if done manually, considering the volume of input data, and the factors needed to be taken into account. Most conferences do the entire assignment process by hand, in spreadsheets, along a malleable set of priorities. This usually means playing with the synthesis of the committees until all delegations desired by the management can be offered a place to the conference close to their preferences.

This process can be automated, if the management composes a definitive set of priorities on how to assign already fixed committee seats — country-committee pairs — to the applied delegates as individuals. The problem of creating the assignments with the aforementioned conditions is a fundamental problem of combinatorial optimization: it is called \emph{maximum weighted bipartite matching} or \emph{assignment problem}~\cite{ropi}. Mathematically speaking, we are looking for the maximum-size matching in a weighted bipartite graph, where the sum of the edges' weight is maximal. One part of the graph is the set of applicants, and the other part is the set of committee seats. The edges between parts represent possible assignments, and their weights are \textquote{goodness} values based on assignment priorities composed by the management. Since this approach does not cover the possible requirement of all delegates from the same school needing to represent the same country, it is not always satisfactory.

If same-school students need to represent the same country, the process can still be automated with maximum weighted bipartite matching, but with an additional constraint: the management must restrict the application to fixed-size delegations. In this case, a participant school's fixed-size delegation applies to the conference as a whole, specifying their preferences from the set of same-size delegations of represented countries on the conference. This way, the bipartite graph's two parts consist of school delegations applied and country delegations represented on the conference. An edge between the two parts means that the two kinds of delegation are of the same size, so they can be assigned to each other. The weight of an edge is the same \textquote{goodness} value based on assignment priorities composed by the management. This approach needs a second application step: after applied school delegations were assigned country delegations represented on the conference, the applicants of the delegations need to distribute the assigned committee seats among themselves.

\subsubsection{Payment}

The participation fees are collected after the registration, generally in multiple parts. International bank transfers executed by students and teachers usually renders the work of the conference's accounting department challenging, as it is often difficult to identify which delegation made which payment. Due to the general lack of integrated payment solutions in the MUN scene, conferences often face a heavy burden of financial paperwork.

A conference management system with integrated payment processing and accounting features could solve these organizational problems. It would also provide a cleaner experience to the participants: they could pay instantly and individually, without the troubles of needing to transfer the money in groups, as whole delegations.

\subsubsection{Travel, hotel, meal and other arrangements for participants}

With the closure of the acceptance and the assignments procedure, the final list of students and their accompanying teachers attending the conference becomes available, and the management can start working on participants' personalized experience. Though habits differ, most conferences offer various convenience services for additional fees, such as transport within the city, accomodation, meal arrangements, and tailored sightseeing or other entertainment programs. This causes lots of additional paperwork with regards to the delegation's arrival and departure dates, hotel or other kinds of accomodation preferences, and several kinds of meal allergies and eating habits\footnote{Since participants arrive from all around the world, conferences generally offer multiple types of menus respecting various cultures and allergies.}.

Similarly to the registration, most of the organizational paperwork in connection with convenience services is done manually. A capable software system possessing all information of a delegation's precise schedule, as well as their hotel, meal and entertainment preferences, could automate all this paperwork. This would allow the management to focus on the quality of the provided services instead of administration. Also, it would open additional financial possibilities. Following preliminary arrangements between the conference management and hotels or other service providers, the applicants could purchase their necessities by selecting them from a list of integrated services during the application procedure. This would endorse further financial cooperation between parties — the conference, the company providing conference management software, and the service providers —, while providing a simple, one-step payment flow for the participants.

\subsubsection{Personalized conference accessories}

MUN conferences formally require participant students, teachers, and all other conference personnel to possess various personalized items. Every person — participants, guests, organizers, staff — should wear an official, conference-issued badge during all events of the conference. The badge usually describes the person's name and title — either a diplomatic one within the simulation in case of participants or guests, or an organizational one in case of organizers or members of staff. Participant delegates, chairpersons, teachers, guests and everyone else attending committee sessions should have official, conference-issued placards in front of them, describing their represented country or position within the committee. Placards are used for voting in larger committees, as well as for identifying delegates from the distance.

Producing hundreds or thousands of personalized items without experience and proper tools can be time-consuming. Management teams generally possess neither the experience nor the tools, which leads to hours or days of manual work of creating personalized elements, one by one. Besides being error-prone, this process consumes valuable organizational resources. A capable software possessing all necessary information about participants and organizers could automate this manual work.

\subsubsection{Conducting the conference}

Following a half-year or longer period consisting of planning and preparation only, conducting the conference itself is usually not a real challenge. Administratively, the management needs to record whether incoming delegations signed in, and received their conference packages, but from the opening ceremony through the committee sessions and parties to the closing ceremony, the conference generally goes by itself, being already scheduled well.

\subsubsection{Processing and storing resolutions and conference data}

Few MUN conferences pay attention to the collection and storage of conference data. As management teams usually lack the resources and experience for keeping digital or paper-based documents, produced resolutions and other conference data is generally not preserved. A capable conference management system could offer solutions for this: with online resolution editing and automated video recording features, the management would not even need to collect conference data, because all of it would immediately be saved in the conference management software's permanent online database.

\subsection{Personal experience: Budapest International Model United Nations}
\label{section:personalexperience}

The previous section's points are mostly based on my personal organizational experience regarding Model United Nations. My former high school, József Eötvös Secondary Grammar School, located in the 5\textsuperscript{th} district of Budapest, first organized \emph{Budapest International Model United Nations (BIMUN)} in April 2011.\footnote{The school has organized BIMUN conferences every year since then. Unfortunately, the tenth anniversary session of BIMUN, which would have been held this year, got cancelled due to the COVID-19 pandemic.} BIMUN was among the first large-scale international MUN conferences in Hungary~\cite{bimunhistory}. Previously I had attended several foreign MUN conferences, and I was looking forward to taking part in the organizational process of an international conference welcoming hundreds of students and diplomatic guests. Over the course of 6 years, I fulfilled several positions: I was a photographer, team lead, deputy Secretary-General, and part of the chief management as an executive advisor.

\section{Social networks}

\subsection{Introduction}

The term social network is used in social sciences, denoting a network of linked individuals or organizations, connected by social relations and interactions~\cite{Borgatti892}. This thesis focuses on another aspect of social networks: online software providing networking, information sharing and messaging features for participating individuals or organizations. The two interpretations are related: users of a social network software form a social network in the scientific sense, and a social network can be loaded into a social network software for analysing patterns, or simply for facilitating interactions among social actors. Analysing social networks can be useful in various fields, including organizational studies and information sciences, and also in diplomacy~\cite{networkdiplomacy}.

\subsection{Connection with graph theory}

Since social networks essentially are connected entities, an obvious choice for modeling such networks is using graphs, where vertices are individuals or organizations, and edges are connections or interactions between them. This way networks are easier to visualize and understand~\cite{socialnetworkvisualization}, and graph algorithms are useful for various kinds of analyses~\cite{socialnetworkanalysis}. By using different kinds of edges in the same graph, several kinds of connections can be represented within the same entity set, allowing to build a multi-dimensional model of the network. This can lead to deeper insight into network structure.

\subsection{Social networking services}

In this thesis I consider social networking services as social networks with associated features the network members can utilize, offered as mass-available services on the Internet. Besides the core networking activity of building relationships within the network itself, associated features can include messaging, information sharing, and collaborational functionality, among others. Usually the goal of a social network is to provide the possibility of interaction among participants, beyond actual in-person interactions. This enables individuals to connect with others regardless of physical distance.

Social networks can be categorized into four main types~\cite{thelwall2009}, but with regards to the focus of this thesis, I divide the set of social networks into two parts.

\begin{itemize}
\item \emph{Generic social networks} have no special characteristics or determined target audience: they offer a set of generic features for people to build and maintain virtual relationships with or without personal acquaintance.
\item \emph{Domain-specific social networks} have a determined target audience: they offer specialized features for members of the target audience in line with specific goals of usage in a given domain.
\end{itemize}

Real-world examples for generic social networks include Facebook~\cite{aboutfacebook}, Twitter~\cite{abouttwitter}, and Instagram~\cite{aboutinstagram}. Facebook is the biggest social networking service in the world, having approximately 2.6 billion monthly active users~\cite{facebook2020q1report}. It has several kinds of features for sharing ideas, photos, live or recorded videos, connecting with friends, and informing others in various ways. It also offers basic event-handling capabilities for facilitating in-person social interactions. Facebook's business model is based on targeted, personalized advertisement: it offers free features for users, who receive ads in their newsfeed and on other interfaces~\cite{fb-business-model}. The platform has been continuously learning user preferences, and personalizes ads according to the user's assumed needs. Facebook's financial income is mostly based on the fees paid by businesses for publishing and targeting their advertisements in various applications of the social networking platform.

Twitter is a social network that gained popularity by providing microblogging features. Users can publish short \textquote{tweets}: posts with at most 280 characters.\footnote{Until 2017 November, the maximum length of a tweet was 140 characters~\cite{twitter-doubling-character-limit}.} It is asymmetric: users can follow other users instead of needing to get virtually acquainted, although for private profiles, users need to be explicitly granted access. Twitter has approximately 166 million daily active users~\cite{twitter2020q1report}, and builds its revenue upon advertising and data licensing~\cite{twitter-business-model}.

Instagram is a social network primarily used for sharing photographs capturing important moments. Despite the fact that it is mainly used for image sharing, Instagram can still be regarded as a generic social network, because it is not to be interpreted within a specific domain, and has no determined target audience. Similarly to Twitter, it is asymmetric, but also has private profiles. Instagram was acquired by Facebook in 2012~\cite{facebookacquiresinstagram}, and shares the same basic characteristics regarding its business model: users receive sponsored posts and stories as personalized advertisements.

Diplomatiq is a domain-specific social network. Its domain is diplomacy, its target audience is the set of junior and senior diplomats and other people working in diplomacy, and its specialized features — currently a very basic set of functionalities — include organizing MUN conferences. Broadening the capabilities of Diplomatiq will expectedly not alter, only broaden its domain-specificity.

\section{Examples for domain-specific social networks}

As previously described in \Cref{chapter:preliminaries}, this thesis regards domain-specific social networks as networks having a determined target audience, offering specialized features for members of the target audience in line with specific goals of usage in a given domain. Real-world examples for domain-specific social networks include LinkedIn\footnote{https://www.linkedin.com}~\cite{about-linkedin}, DeviantArt\footnote{https://www.deviantart.com}~\cite{about-deviantart}, and also Diplomatiq\footnote{https://app.diplomatiq.org}.

LinkedIn is an employment-oriented social network, facilitating professional networking: a connection between profiles usually represent real-world professional relationships. Having nearly 690 million registered members~\cite{about-linkedin}, it is the largest network of its kind. LinkedIn is a domain-specific network: its domain is human resources and employment, and its main target audience is job seekers and employers or recruiters. Besides general social network features like content sharing and messaging, job seekers can create professional profiles similar to a curriculum vitae, find jobs by specific criteria, and apply to job opportunities on the platform. Employers can list job opportunities on LinkedIn, while recruiters can interact with job applicants in the form of formalized or personalized messages. One interesting feature of LinkedIn's social network is \emph{endorsements}: people can positively endorse each other's skills, allowing recruiters to select applicants based on community-approved claims, besides their education, previous work experience, and personally listed skills. LinkedIn's revenue is primarily based on a subscription model for its premium recruiting tool, which allows recruiters to access additional data about job seekers otherwise not available on the platform~\cite{linkedin-business-model}. Apart from its recruiting tool, LinkedIn generates revenue from personalized advertisement as well.

DeviantArt is a domain-specific social network for sharing several kinds of artwork. Its domain is art and its target audience is artists and art lovers. Its features support sharing artwork of traditional and digital drawing and painting, photography, literature, filmmaking, among others. DeviantArt has over 45 million monthly unique visitors~\cite{about-deviantart}. It generates its revenue by advertising, offering subscription-based memberships, brand partnerships, and producing prints for artwork published on the platform~\cite{deviantart-revenue}.

\section{MyMUN}
\label{section:mymun}

\subsection{Introduction}

MyMUN is a domain-specific social network, claiming to be \textquote{the ultimate MUN database, conference management tool, and social network}~\cite{mymunwebsite}. In the short run, it is a potential competitor of Diplomatiq.\footnote{Diplomatiq will outgrow the world of Model United Nations in the long run, which means it cannot be collated with MyMUN, as the two services will operate over different domains.} Its domain is Model United Nations, its target audience is organizers and participants of MUN conferences, and it provides functionality both for organizing an MUN conference, and for preparing to one as a participant delegate.

MyMUN's website does not contain detailed company data, but according to other sources, the company was founded in 2014~\cite{mymunfacebook} and seems to be an established startup, based on its 17 employees and \$3 million of annual revenue~\cite{mymunzoominfo}. From this point forward, every information disclosed in \Cref{section:mymun} originates from my personal experience of using MyMUN via its website~\cite{mymunwebsite}, therefore I disregard references until the end of this section.

\subsection{Organizational features}

The platform offers a number of features for making the conference organization easier. However, having tried out the software in a variety of different scenarios, I experienced the application to frequently produce failures regarding even its basic functionality.\footnote{As the website does not list any means of contact for feedbacks about the software, and I did not find any possibility to contribute to the closed-source MyMUN project, I have contacted the Chief Technology Officer via email, with a list of experienced failures and detailed reproduction steps.}

\subsubsection{Registering an MUN society and a conference}

In MyMUN, the first step of organizing an MUN conference is to register an organization, called \textquote{MUN society}. The user registering the MUN society becomes the sole administrator of the organization, and the only administrator of all MUN conferences hosted by the organization. Even though further organizer members can be invited to administer a conference, no access control is offered by the platform: invited organizers have the same permissions as an administrator. Since MUN organizational staff usually rotates quickly, and transferring the ownership of an MUN society or conference is not possible on MyMUN, this approach is not suitable for conferences hosting sessions over multiple years.

An MUN society can host multiple \emph{conferences}. For registering a conference in the system, the user needs to supply the conference's name (e.g.\ Budapest International Model United Nations) and codename (e.g.\ BIMUN), number of expected delegates on a session, contact information, and the start and end dates of the conference. Sadly, the platform does not allow the conference to have \emph{sessions} as child entities, which means that a new conference entity needs to be registered for sessions of the same conference in different years. This excludes the possibility of e.g.\ tracking participants' performance on the same conference over different sessions, because there is no entity in the data model to connect the different sessions stored as separate conference instances. After submitting the necessary information, the conference gets immediately listed in the Discover section, allowing delegates and other participants to apply.

\subsubsection{Conference application procedure}

MyMUN offers various settings for the application procedure. Additional application data can be queried from applicants in the form of short text, long text, numeric, multiple choice, checkbox or file inputs. The organizers can manage the application of individual delegates, delegations, committee chairpersons, observers, and faculty advisors. The platform does not offer creating and managing custom roles beyond the previously listed ones, but each role's application deadlines and additional questions can be adjusted separately. Even though applications can be accepted or rejected, the system does not offer an automated way to accept or reject applications with regards to the best combination of applicants and country-committee assignments based on priorities. In MyMUN, a participant's application is either accepted or rejected, and the question of the applicant's country-committee assignment is addressed in a later organizational phase.

\subsubsection{Committee and country assignments}

The platform uses the term \emph{Country Matrix} for denoting a matrix having committees on the X axis, and countries (committee seats) on the Y axis\footnote{The term \emph{country matrix} is not specific to MyMUN, it is a common term in the MUN scene.}. In the Country Matrix, organizers can assign accepted applicants to seats, or accepted applicant delegations to country delegations, if their sizes allow. The \emph{Assignment Wizard} provides an automated way to assign committee seats to most of the delegates and delegations based on their preferences provided at the application, but the wizard cannot be configured to take into account additional priorities or restrictions. And even if it could, it still operates on the set of already applied participants, disallowing acceptance with regards to the best combination.

\subsubsection{Handling finances}

The financials of conference participation fees can be managed with MyMUN in various ways. Payment settings enable to adjust fees for different payment classes of different kinds of participants separately — delegates, head delegates, committee chairpersons, faculty advisors, and delegations —, but it is not possible to add custom payment classes. Participants can pay individually or in groups, with credit/debit cards, PayPal, or bank transfer, directly through MyMUN. The platform allows to issue refunds as well. Also, promotions and coupons can be offered to participants.

The organizational dashboard details a financial summary with all received and refunded payments broken down to payment classes. The cash flow view shows the financial reserves and the predicted income of the conference. Payment history can be exported with financial identifiers, so accountants of the conference can easily process payments.

\subsubsection{Automated notifications and reminders}

Relevant events of the application process are confirmed via automated email notifications. The confirmed events are the submission of an application, the acceptance or rejection of an application, and the assignment of a committee seat to an accepted applicant. MyMUN offers basic email notification templates for these events, which can be completed with additional HTML or text content in the settings.

\subsubsection{Statistics}

Organizers can create custom views of conference data stored in MyMUN. Custom views can include several metrics based on application type, application status, and other characteristics. Displaying such customized data can help organizers to decide if the conference met predefined numeric goals, e.g.\ diversity, distribution of genders in committees, etc.

\subsubsection{Advertisement and marketing}

Although most conferences usually have their already established audience\footnote{With proper networking skills and some personal popularity, there is usually no need for significant marketing efforts. It is usual that organizers organically pre-promote their conferences in the MUN community, either in-person on other events, or on social media. On the very first session of BIMUN in 2011, the number of applicants was five times the conference capacity, as organizers put serious efforts into organic marketing among their friends and acquaintances.}, some need more than organic reach only. MyMUN provides a variety of promotional packages for boosting the visibility of an event. Packages include advertising on the platform itself by highlighting the conference on one of the promotional pages, on MyMUN's social media accounts, and through email campaigns. Packages can be optimized, and one can also compose a fully customized marketing campaign of the individual promotional items.

\subsection{Features for participants}

\subsubsection{Listing MUN conferences}

Students can browse among conferences in the \emph{Discover} section of MyMUN. This section consists of several views: the main view is a page highlighting upcoming and featured conferences. Events are categorized by continent, and featured conferences are displayed at the top of the page, and across categories. Besides the highlights view, other views display the conferences on a map, in a calendar, or in a list view providing filtering capabilities.

All views allow visitors to apply to displayed conferences. The most complete display of events stored in the system is in the regular list view, where also the most information is displayed of individual conferences at once. Conferences can be ordered by name, date, location, number of delegates, fee, and rating.

\subsubsection{Applying to a conference}

The application procedure starts with checking and modifying one's personal data as required. This step ensures that the application is performed with up-to-date information, in case any personal identifiers or characteristics — e.g.\ diet or allergies — stored in the system became obsolete. Data saved during this step is also updated globally.

In the second step, the applicant chooses their role on the conference. This can be delegate, member of a delegation, chairperson, faculty advisor or observer. Some of the roles cannot be selected if preliminary conditions are not met, e.g.\ users with the occupation student cannot apply as a faculty advisor. If the applicant chooses to be a member of a delegation, then they need to supply its name, and then the delegation's head delegate needs to confirm the membership in the system.

The third step lists the terms and conditions of a conference. Having contractual arrangement between conferences and participants is not ordinary, therefore this step is included only if the conference formulated and submitted such documents to MyMUN. If there are any, accepting the conditions is a requirement of the application.

The applicant needs to supply their committee seat preferences in the fourth step, first choosing the committee from the set of the committees offered by the conference, then the represented country within the committee with a search-assisted drop-down menu\footnote{A search-assisted drop-down menu allows users to choose a value from a set, while providing search capabilities in the set.}. The minimum and maximum number of necessary assignment preferences is conference-specific. As described before, the software's \emph{Assignment Wizard} feature claims to assign committee seats to applicants closest to their preferences overall.

Additional conference-specific questions need to be answered in the fifth step. Most conferences ask about applicants' personality, MUN society membership and previous MUN experience. Usually the goal of additional questions is to evaluate an applicant's attitude and commitment. Larger conferences implement sophisticated questionnaire often requiring serious professional background to fully answer.

The sixth step requires the applicant to formulate a motivational letter in which they introduce themselves and describe their personal and professional interest in the conference. The answer's input field supports long text formatting with optional images, links and other kinds of attachments. A separate document acting as the motivational letter itself cannot be uploaded, meaning that applicants need to prepare this document online, although it is automatically saved during editing.

In the seventh and final step applicants can review their application before submission. All steps' information can be separately edited, navigating the process back to the step to be amended. The application procedure can be either finished by submitting the application or aborted by withdrawing the application and deleting all saved data.

\subsubsection{Offers during the application process}

MyMUN offers several kinds of services already during the application process. In a continuously visible separate sidebar, hotel, flight and health insurance offers are shown. Hotels, motels and other accomodation opportunities are displayed on an interactive map covering the vicinity of the conference, indicating the event's exact location. Selecting an accomodation option reveals its price and community rating, and the offer can be reserved instantly, after the user was navigated to the accomodation service provider's website. Besides generic offers, conference-specific contractual accomodation offers can be displayed in a highlighted manner. Flights and other means of travel are also directly offered to applicants by integrating external service providers into MyMUN. Besides travel, the platform offers conference-specific health insurances as well.

\subsubsection{Research and study guides}

Model United Nations is an academic activity, and thus it involves conducting research before attending a conference. MyMUN offers a virtual library of study guides and position papers\footnote{An MUN position paper summarizes the position of the delegate's country regarding a given topic or issue. Its purpose is to prove the delegate's or delegation's level of preparation before the conference, and to serve as a topical fact sheet during the event.}, which will be extended with a section for drafted resolutions as well. The platform's position paper database counts over 14,000 entries. MyMUN claims that it will also serve online courses on Model United Nations in general, and specifically on preparing to conferences as a delegate.

\subsubsection{Travel services}

MyMUN promotes and cross-sells several kinds of travel services, some in cooperation with conferences. Services include flights, accomodation, health insurance, car rental, and adventure tours. A so-called \emph{group service} offers a package of integrated services tailored to delegations: after receiving all data from the head delegate or a faculty advisor, MyMUN acts as a travel agency, and makes reservations for flights and accomodation, books required travel insurances, and sends a detailed travel offer to the delegation.

\subsection{Business model}

The financial basis of MyMUN lies on paid conference organization features, conference advertisement and marketing, the cross-sales of travel services, and on a three-level subscription-based membership in the \emph{Delegates Club}, a closed circle of MUN participants offering exclusive content, savings and insurance benefits.

\subsubsection{Paid conference organization features}

MyMUN offers two plans for conference organizers. In the free \emph{Trial} plan, basic features are available, like listing a conference on the platform, adding committees and organizers, and the management of chairpersons' application. The Trial plan is not sufficient for assisting in the organization of an MUN conference from the beginning to the end, encouraging organizers to subscribe to the \emph{Professional} plan, which allows to utilize all features of the platform. The Professional plan costs 4\% of the value of each financial transaction actuated through MyMUN's payment system, but minimally €4 per transaction. It is invoiced either directly to delegates, integrated to the online payment procedure of a conference application, or to the conference, if online payments are not available in a given area of operation.

\subsubsection{Conference advertisement and marketing}

As described previously, MyMUN allows organizers to promote their conferences in various ways. The offered packages containing several kinds of promotions are priced between €300 and €3,000. The fee of individual marketing elements — like featuring a conference on chosen landing pages or in email campaigns — vary between €100 and €1,500.

\subsubsection{Cross-sales of travel services}

Although directly not mentioned, fees of offered travel services probably contain dividends received by MyMUN. It is a common sales scheme, when a seller offers a service, and a second party offers a large audience in assumed need of the offered service, in exchange for dividend. MyMUN's offerings — flight tickets, accomodation, health insurance and group services — all provide ways for selling the services with considerable financial benefits.

\subsubsection{Delegate's Club}

The Delegate's Club is a subscription-based membership group within MyMUN, offering three levels of additional services for conference participants. Membership is not a requirement for using the platform for conference applications, but it comes with benefits. The smallest \emph{Bronze} package costs €27 monthly for 6 months: it offers a flatrate for health insurances purchased through MyMUN, and full access to the position paper database. The \emph{Silver} package costs €65 monthly for 1 year, includes all benefits of the Bronze package, and comes with a free ISIC Card\footnote{The ISIC Card is a student ID card, the only one of its kind which is internationally accepted.} offering discounts for hotels, hostels, restaurants and museums. In addition to all benefits of the Silver package, the \emph{Gold} package offers discounted subscription to diplomatic periodicals, and costs €90 per month for a fixed 1-year term.

\section{MUNPlanet}

MUNPlanet was the largest MUN community in the world in the form of an online knowledge network where MUNers create, curate and share their knowledge and experiences about issues of global importance~\cite{munplanetfacebook}. It was a potential competitor of Diplomatiq, being a global, domain-specific social network expanding along Model United Nations. Its domain was Model United Nations, its target audience was all participants of MUN conferences, and its specific features were to facilitate knowledge sharing within the domain.

No further information is available of MUNPlanet, since at the time of writing this thesis, its website~\cite{munplanetwebsite} is not reachable, and its Facebook page~\cite{munplanetfacebook} having more than 150,000 followers received its last update in June 2019.
